\subsection{\linux Directories overview}

The organization of a \linux file system is completely defined by the
\emph{Filesystem Hierarchy Standard}.

It is defining the notion of \emph{prefix}. A prefix is a directory
that will contain the installation of one or more software,
libraries\ldots
It contains several standardized sub-directories such as:
\begin{description}
  \item[\texttt{bin}] for executable binaries (i.e. applications),
  \item[\texttt{etc}] for editable configuration files,
  \item[\texttt{include}] for header files (C/C++ specific).
  \item[\texttt{lib}] for static and shared libraries,
  \item[\texttt{share}] for resources. In particular, documentation is\
    stored in \texttt{share/doc}.
\end{description}

Some standard prefixes are managed by the system:
\begin{description}
\item[\texttt{/}] Root prefix, containt low-level software required
  for computer boot.
\item[\texttt{/usr}] stands for Unix System Resources, contains
  software managed by the local packaging system.
\item[\texttt{/usr/games}] for games.
\item[\texttt{/usr/local}] is the default prefix for manually managed
  software. It means that an automatic packaging system will
  \emph{never} modify the content of this directory.
\end{description}

These prefixes are shared: any user can access to them with the
exception of the \texttt{/usr/games} prefix which is owned by the
\texttt{games} group. It means that one has to be part of this group
to run a game.


Being a prefix is an absolutely arbitrary and symbolic decisions:
there is no list of prefixes on the system and a prefix is a
``normal'' directory. Only its structure can indicate that a directory
is a prefix.

It also means that a user can create a new prefix to install some
packages separately. Several reasons can lead to this choice such as
having the possibility to install development or unstable version of
some software. As no non-shared prefix is available by default, users
who do not have root access to their machine have to setup a local
prefix in their home directory to root access.


\subsection{Adding a prefix manually}

Creating a prefix is just adding some directories. Int this section, a
prefix will be created in the \texttt{/my/new/prefix} directory. Feel
free to replace it by whatever you want in each command.

\begin{verbatim}
$ mkdir -p /my/new/prefix  && \
  mkdir /my/new/prefix/bin && \
  mkdir /my/new/prefix/lib
\end{verbatim}

Then, fix directories permissions to give access to the users and
groups you want. Read-only access is sufficient to use software, write
access has to be given for people who will be able to manage software
in this prefix. Directories also have to be traverseable.


Additional steps are required so installed files will be found xby
system tools.

An environment variable is a variable which is passed to the shell
sub-processes (typically the application you want to launch).  Some of
them have special meanings and control the system behavior.

In particular the \texttt{PATH} environment variable contains the list
of the directories where binaries are stored. The binary directory of
the new prefix has to be prepended to the variable.

For the same reason, \texttt{LIBRARY\_PATH} and
\texttt{LD\_LIBRARY\_PATH} have to be updated with the new library
directories. The first one is used by the C/C++ compilers to locate
libraries whether the second is used to retrieve libraries at run-time
when an application is launched. \texttt{C\_INCLUDE\_PATH} adds the
\texttt{include} directory to the list of default header files.

\texttt{INFOPATH} and \texttt{MANPATH} are updated to indicate the new
documentation directories.

The environment variables modifications can be done through the
following command lines:

\begin{verbatim}
# If you use a Bourne shell:
$ export C_INCLUDE_PATH="/my/new/prefix/include:${C_INCLUDE_PATH}"
$ export LD_LIBRARY_PATH="/my/new/prefix/lib:${LD_LIBRARY_PATH}"
$ export LIBRARY_PATH="/my/new/prefix/lib:${LIBRARY_PATH}"
$ export INFOPATH="/my/new/prefix/share/info:${INFOPATH}"
$ export MANPATH="/my/new/prefix/share/man:${MANPATH}"
$ export PATH="/my/new/prefix/bin:${PATH}"

# If you use a C-shell:
$ setenv C_INCLUDE_PATH "/my/new/prefix/include:${C_INCLUDE_PATH}"
$ setenv LD_LIBRARY_PATH "/my/new/prefix/lib:${LD_LIBRARY_PATH}"
$ setenv LDPATH "/my/new/prefix/lib:${LIBRARY_PATH}"
$ setenv INFOPATH "/my/new/prefix/share/info:${INFOPATH}"
$ setenv MANPATH "/my/new/prefix/share/man:${MANPATH}"
$ setenv PATH "/my/new/prefix/bin:${PATH}"
\end{verbatim}

However, this commands do not change the value of an environment
variable permanently, it only change it in your local shell
session. To make the modification global, you have to add them to your
own shell configuration file (see your shell documentation for more
information).

To finish, keep in mind that the order in this directory list is
important: a list is parsed from left to right. It means that if one
tries to start a program called \texttt{foo}, the leftest prefix
containing that application will be used.


\subsection{Using distribution-specific packages (source or binary)}


To use a package, one has to launch the tool that are provided with
their \linux distribution. Adding or removing files in a managed
prefix can break permanently a system.


The packaging systems currently provide command line tools and graphic
tools. For instance on \ubuntu, the graphical tool is called
\texttt{synaptics} whereas the command line tools are \texttt{apt-*}
tools (\texttt{apt-get}, \texttt{apt-cache}, \ldots).


\subsection{Using tarballs}


Tarballs are much more difficult to install than packages as the
process has to be done manually.  The next paragraphs will describe
step by step how to use a tarball.


\subsubsection{Obtaining a tarball}

Most of the projects are providing a tarball using the
\texttt{.tar.bz2} or \texttt{.tar.gz} formats. Download it on the
project web page.

They sometimes provide a hashsum of the file to make sure no error
happened. It is recommended to check that the hashsum matches the
downloaded file.


\subsubsection{Uncompressing a tarball}


A tarball is a \textbf{compressed archive}. It is created by two
separate steps: archiving and compression.

Archiving is the process of taking several files and directories and
put them in a single file. This file is called an archive. The most
common archiver is \texttt{tar} and produced \texttt{tar} files
(i.e. \texttt{my-software.tar}\footnote{and not
  \texttt{my-software.tar.something}!}).

The compression is a second process which takes one or several files
and then compress it/them. Two popular algorithms and gzip and bzip
producing respectively \texttt{gz} files and \texttt{bzip} files.

Indeed, a \texttt{.tar.bz2} file is an archive that has been
compressed using the bzip algorithm.


To uncompress and unarchive a file, one can type:
\begin{verbatim}
# For a bzipped tarball:
$ tar xjvf my-software.tar.bz2
# For a gzipped tarball:
$ tar xzvf my-software.tar.gz
\end{verbatim}

A good and widely respected practice under \linux is that tarballs
contains a single directory with the same name as the archive
containing its whole content. It avoids mixing the tarball's contents
with the user's files.


\subsubsection{Configuring}


Once the tarball is uncompressed, user has to setup the package by
launching the \texttt{configure} script shell.

A wise idea is reading the \texttt{INSTALL}, \texttt{README} and
\texttt{NEWS} files to read about specific issues and last
modification of the software.

Then \texttt{./configure --help} will display the list of options for
the package. The standard options will be discussed here but specific
ones such as optional features will differ from pacakge to package.


It is generally a good idea to compile out of source. It means that
generated files will be stored in a separate directory instead of the
source directories. By doing it, one just have to change its build
directory to restart from scratch without error. Generated files which
are not deleted properly are the source of many compilation errors.

To build out of source, just choose a directory and call the
\texttt{configure} script from it. For instance, one can create a
\texttt{\_build} directory at the root-level of the uncompressed
package:

\begin{verbatim}
# To be runned in the package root directory:
$ mkdir _build && cd _build && ../configure
\end{verbatim}


However, one might want to customize the package, it can be done
through options or environment variables.


Several standard options are provided:
\begin{description}
\item[\texttt{--prefix}] The prefix that will be used for installation
\item[\texttt{--XXXdir}] Binary, resources directories\ldots can also
  be defined by hand instead of being its default prefix
  sub-directory.
\item[\texttt{--program-prefix}, \texttt{--program-suffix},
  \texttt{--program-transform-name}] are useful options used to
  transform program name. It is used to install several version of the
  same program without any conflict.
\item[\texttt{--host}] can be used for cross-compiling (i.e. compiling
  for a platform different from the computer running the compilation).
\end{description}


The environment variables control the behavior of the compiler. Here
is some tips for C/C++ packages which represent the larger part of the
software packages:

\begin{description}
\item[\texttt{CC}] defines the C compiler.
\item[\texttt{CXX}] defines the C++ compiler.
\item[\texttt{CPPFLAGS}] defines the flags that will be passed to the
  C/C++ pre-processor.
\item[\texttt{CFLAGS}] defines the flags that will be passed to the C
  compiler.
\item[\texttt{CXXFLAGS}] defines the flags that will be passed to the
  C++ compiler.
\item[\texttt{LDFLAGS}] defines the flags that will be passed to the
  linker.
\end{description}

Flags passed to the pre-processor and to the linker are usually only
custom header and libraries locations. On the opposite,
\texttt{CFLAGS} and \texttt{CXXFLAGS} controls how the compiler will
build the binaries.

Compilers accept a huge set of options which will not be described
here. Please consult your compiler documentation for the best set of
options depending on your goal (debug, release, packaging) and
platform. Some general purpose compilation profiles are available in
\autoref{compilation_profiles}.


\subsubsection{Building and installing}


Once the package is configured, you build directory contains
\texttt{Makefile} files that can be used to build the software.
Only one command line is required:

\begin{verbatim}
$ make
\end{verbatim}


It starts the package compilation process which can take up to several
hours. When it is finished, you can install the package through the
following command:

\begin{verbatim}
$ make install
# ...or if install is done in a directory the current user can not write in:
$ sudo make install
\end{verbatim}

It is important to \emph{not} launch the first \emph{make} command as
a super-user: it is unsafe (a bad or malicious package could damage
your system), unneeded and create object files owned by the root user
that will disturb your work if you have to re-use your build
directory.


To check the installation step without root access, one can install in a sandbox by using:
\begin{verbatim}
$ make install DESTDIR=/my/temporary/prefix
\end{verbatim}

It will simulate that the \texttt{DESTDIR} directory is the root
directory. It will create the path to your prefix \textbf{inside} and
install everything. If your prefix is \texttt{foo}, your binary
directory \texttt{foo/bin} and you are building a programm called
\texttt{bar}, it will install it in
\texttt{/my/temporary/prefix/foo/bin/bar}.  This feature is
interesting when building a binary package: one configure the software
using the prefix wanted by the packaging system (usually
\texttt{/usr}), compile and install the whole package in a
sandbox. You then get a snapshot of the build package which is very
easy to transform into a full binary package.
