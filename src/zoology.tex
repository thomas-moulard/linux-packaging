% Tiks styles and settings.
\tikzset{
  pkgstate/.style={
    % The shape:
    rectangle,
    % The size:
    minimum size=6mm,
    % The border:
    very thick,
    draw=red!50!black!50,         % 50% red and 50% black,
                                  % and that mixed with 50% white
    % The filling:
    top color=white,              % a shading that is white at the top...
    bottom color=red!50!black!20, % and something else at the bottom
    % Font
    font=\itshape
  },
  pkgaction/.style={
    % The shape:
    rounded rectangle,
    minimum size=6mm,
    % The rest
    very thick,draw=black!50,
    top color=white,bottom color=black!20,
    font=\ttfamily},
  skip loop/.style={to path={-- ++(0,#1) -| (\tikztotarget)}}
}
\tikzset{pkgaction/.append style={text height=1.5ex,text depth=.25ex}}
\tikzset{pkgstate/.append style={text height=1.5ex,text depth=.25ex}}



To understand how packaging works, one has first to think about how
the work is shared between software developpers and \linux
distributions teams.  Software developpers are writing a specific
piece of software: they want it to be as broadly available as possible
(on different \os and on differents architectures). They will be
choosing develoment tools such as a compiler that will limit the
portability of their application but no other knowledge of their user
\os is available for them. Developpers will just make code as generic
and customizable as possible in order to make it work everywhere.  On
the opposite, maintainers are writing packages for a specific \linux
distribution: they do not target other systems and their distribution
is usually providing support for a limited number of computer
architectures. They know perfectly well how is organized their users
system and want to make all the software ecosystem working together as
best as possible.


This is why there exists two kinds of packages under \linux:
\begin{description}
\item[Tarballs] provided by software developpers,
\item[Distribution specific packages] provided by \linux distribution.
\end{description}


The second category of packages is always generated from te first: a
package is the tarball plus some additional tweaks, fine tuning to
adapt an application to a distribution and meta-information such as
the software name, author names\ldots

Packages can then be divided in two categories: source pacakges and
binary packages. Source packages has to be processed to be useful, it
contains sources files that can be used to generate the application
but not the application itself. On the opposite a binary package
directly contain the application and no further processing is
required.


For a given software a tarball is always available: it is the usual
way of publishing an application. On the opposite, providing a package
needs an additional effort which is not always made. This work is
usually done by volunteers that appreciate a project and thus
encourage its diffusion on their own distribution. Software popularity
is usually related to the number and quality of packages available.

\begin{figure}[htbp]
\centering
\begin{tikzpicture}[point/.style={coordinate},>=stealth',thick,draw=black!50,
                    tip/.style={->,shorten >=1pt},every join/.style={rounded corners},
                    hv path/.style={to path={-| (\tikztotarget)}},
                    vh path/.style={to path={|- (\tikztotarget)}}]
  \matrix[column sep=4mm] {
    % First row:
    & & & & & & & \node (pkgu) [pkgaction]    {package};
    & \node (ubuntu) [pkgstate] {Ubuntu package};\\
    % Second row:
    \node (p1) [point] {}; & \node (us)  [pkgstate] {unpublished software}; &
    \node (p2) [point] {}; & \node (rel) [pkgaction]    {release};          &
    \node (p3) [point] {}; & \node (tb)  [pkgstate] {tarball};\\
    % Third row:
    & & & & & & & \node (pkgf) [pkgaction]    {package};
    & \node (fedora)[pkgstate] {Fedora package};\\
  };

  { [start chain]
    \chainin (p1);
    \chainin (us)  [join=by tip];
    \chainin (rel) [join=by tip];
    \chainin (tb)  [join=by tip];
    { [start branch=ubuntu]
      \chainin (pkgu)   [join=by {vh path,tip}];
      \chainin (ubuntu) [join=by tip];
    }
    { [start branch=fedora]
      \chainin (pkgf)   [join=by {vh path,tip}];
      \chainin (fedora) [join=by tip];
    }
  }
\end{tikzpicture}
\caption{Packaging workflow}
\label{fig:software_workflow}
\end{figure}

As illustrated by \autoref{fig:software_workflow}, providing software
specific packages is the last step of the packaging: it needs
additional effort but it is much more convenient for the user: on
modern distributions, a single command-line or click can install,
upgrade or remove the software.


\subsection{Tarballs}

Project development is divided in different steps: those steps can
differ depending on the developpers policies but can be categorized in
three categories:
\begin{description}
\item[Initial development] is the steps where features are added to
  the software. During this phase, it is not recommended for users to
  rely on the application as many crashes are likely to occur.
\item[Stabilization] begins where enough features have been added to
  make the software useful. Developpers stop adding features and try
  to fix bugs and crashes.
\item[Test] is the phase where initial releases are done. They are
  usually named as ``alpha'' (when features are missing) then ``beta''
  (when bugs happens but most of the features are available) and
  finally ``release candidate or RC''. Early adopters then try these
  releases and give feedback to the development team. Depending on the
  software usability, the alpha becomes a beta and then a release
  candidate. Each release candidate is a potential definitive version,
  but if an important problem is detected, then the another RC release
  is done. This process is repeated as long as severe problems occur.
\item[Final release] is the last step. When developpers and users
  agree, a new stable version is released. This is the one that will
  be used by most of the users.
\end{description}

This description is just a rough idea of how release are working and
vary a lot from a project to another. The most important point is that
as long as there is no stable release, no packaging is done.


Software applications are procuded by writing text files called
``source code'' following a particular syntax called ``programming
language'' that are compiled to produce binaries (i.e. applications).

This applications also make use of external non-generated files such
as documentation or images for instance.

To finish, a software project also contain set of rules that describe
how to compile the project and where to install the different files on
the user's system.


A tarball is only a compressed archive containing these elements. Two
widely used formats are ``.tar.bz2'' and ``.tar.gz''.



Regarding end users, installing a software from a tarball is complex:
you will need the tools required to compile the project and some
fine-tuning will have to be made by hand. However, the control on the
software is complete: one can control what features will be enabled or
not, where the software will be installed\ldots Compiling yourself the
application can also increase application speed.


To finish, there is no meta-information in a tarball: one cannot get a
list of installed packages from a tarball and automatic updates are
impossible.


\subsection{Source packages}


Source packages are meta-information added to a tarball to make it
usable by a packaging system.
The added meta-information are typically:
\begin{description}
\item[Package tracking] Human readable package title, description,
  authors list, licensing information and web site URL.
\item[Compile-time dependencies] Tools used by the compilation process
  but not required when the application is runned.
\item[Run-time dependencies] Dependencies that have to remain
  available on the system each time the package is used such as shared
  libraries.
\item[Actions] A list of actions to perform at install, remove,
  updated (such as adding/removing a shortcut to the desktop).
\item[Patches] A set of modifications to perform on the software
  before compiling it in order to fix bugs and tune it for the current
  \linux distribution.
\end{description}


The packaging tool then stores all the meta-information in a
database. It allows automatic installation, update and removing.

One advantage of source packaging is that the user can choose the
installation directory and still has a good control of how the
software will be compiled without doing it manually.  The drawback is
that compile tools have to be installed and the compile time is very
long for complex software (many hours for large desktop applications).


\subsection{Binary packages}


Binary packaging is more and more popular as it is fully automatic: no
control is available for the user but it is very simple to use and
installation is almost instantaneous unlike source packaging. However,
it needs more effort than source packaging and is not always
available.


If source packaging adds meta-information to the tarball (i.e. to the
non-compiled software), a binary package adds meta-information to the
compiled program.

The packaging system install all the files in the same prefix (usually
\texttt{/usr}) so the user can not choose where the
software will be installed. The application is compiled once by the
package maintainer in the pre-determined installation directory.
However, the maintainer installs the application in a sandbox: an
empty temporary directory.  This directory will contain all the files
that would have been copied in the system. The content of the sandbox
directory and the meta-information file are archived and compressed.
This compressed archives is the binary package.


These source or binary packages are then often gathered in
``repositories'' to allow automatic retrieving from the package
management software installed on the user's computer.  A repository
contains a packages list and the corresponding files: each
distribution provide one or more default repositories but custom ones
also exist for minor packages or special versions (unstable or
dedicated to a particular task).


\subsection{Comparison}

The \autoref{fig:package_comparison} illustrates the advantages and
drawbacks of each packaging style. As expected, the tarball is the
most complex way of installing software and is not recommanded to
non-technical users. On the opposite, advanded users are often
interested to use the new versions of their software before the
packaging work is done.


On the opposite, source and binary packages allows easy installation
but as the settings are defined by the maintainers, the user does not
control anymore the fine tuning. For instance, experimental features
that are considered too unstable might be irreversably disabled by the
maintainers to ensure the qualitry of their \linux
distribution. Modifications are also done to better fit the
distribution: the location of configuration files are often changed to
match the distribution standard. It can lead to confusion new users
that read documentation on the web that do no match their system
package.


The last problem is package relocability: a package is said to be
relocable if the end-user can install it anywhere. However, only a few
package system support this: most of them force the installation to
\texttt{/usr}. This directory is only writeable by the
super-user (i.e. root user) which means that package management has to
be done by the computer administrator. If one is not the administrator
of its system, the only choice still remains the installation from the
tarball.

\begin{figure}[htbp]
\centering

\begin{tabular}[htbp]{|c|c|c|c|}
  \hline
  Packaging style & \textbf{Tarball} & \textbf{Source package} & \textbf{Binary package}\\
  \hline
  \emph{Dependency tracking}
  & \textcolor{red}{no}
  & \textcolor{green}{yes}
  & \textcolor{green}{yes}
  \\
  \emph{Enable/disable optional features}
  & \textcolor{green}{yes}
  & \textcolor{green}{yes}
  & \textcolor{red}{no}
  \\
  \emph{Package removing}
  & \textcolor{red}{unsafe}
  & \textcolor{green}{safe}
  & \textcolor{green}{safe}
  \\
  \emph{Installation time}
  & \textcolor{red}{long}
  & \textcolor{red}{long}
  & \textcolor{green}{short}
  \\
  \emph{Automatic package management}
  & \textcolor{red}{no}
  & \textcolor{green}{yes}
  & \textcolor{green}{yes}
  \\
  \emph{Installation choice}
  & \textcolor{green}{yes}
  & \textcolor{green}{yes}
  & \textcolor{red}{no}
  \\
  \emph{Root access}
  & \textcolor{green}{not needed}
  & \textcolor{red}{required}
  & \textcolor{red}{required}
  \\
  \emph{Ease of use}
  & \textcolor{red}{-}
  & \textcolor{green}{+}
  & \textcolor{green}{+}
  \\
  \emph{Package control}
  & \textcolor{green}{++}
  & \textcolor{green}{+}
  & \textcolor{red}{-}
  \\
  \hline
\end{tabular}

\caption{Comparison of packaging methods}
\label{fig:package_comparison}
\end{figure}

