GCC accept a large set of flags and it would be difficult to discuss them all here. On the opposite, there is only a few amount of user categories:
\begin{description}
\item[developpers] or power-users that want to get technical information,,
\item[maintainers] which want to package the software,
\item[end-users] that just want to use the package,
\end{description}

\subsection{Developper profile}

The developper profile:
\begin{itemize}
\item allow easy debugging by adding debugging symbols and not enabling optimization.
\item do not disable assertions.
\end{itemize}

Basically, this is almost completely the default behavior of GCC so very few options are required:
\begin{verbatim}
CFLAGS="-g -O0"
CXXFLAGS="-g -O0"
\end{verbatim}

If one only use GDB as a debugger, \texttt{-g} can be replaced by
\texttt{-ggdb} to enable GDB specific debugging extensions.

\subsection{End-user and maintainer profile}

The end-user profile:
\begin{itemize}
\item want to get the application to run as fast as possible,
\item does not want technical information
\item wants small binaries.
\end{itemize}

\begin{verbatim}
CPPFLAGS="-DNDEBUG"
CFLAGS="-O3"
CXXFLAGS="-O3"
\end{verbatim}

One can also add the \texttt{-march} options to \texttt{CFLAGS} and
\texttt{CXXFLAGS} to build binaries specific to your platform.  For
intance, to generate binaries optimized for Intel Core 2 Duo
processors, one can use \texttt{-march=core2 -msse4.1} (with
\texttt{GCC >= 4.3}).

To find the options matching your processor, see the compiler
documentation. Processor information can be displayed using:
\begin{verbatim}
$ cat /proc/cpuinfo
\end{verbatim}


The maintainer wants almost the same thing than the end-user except
that the binaries are built to be shared. It means processor-specific
flags must be avoided.


To finish, aggressive compilations flags can be used to optimize more
the binaries. Some of these flags are not supported by older versions
of GCC or may trigger problems or bugs:

\begin{verbatim}
CPPFLAGS="-DNDEBUG"
CFLAGS="-O3 -funroll-loops -frerun-loop-opt -fschedule-insns2
        -frerun-cse-after-loop -falign-functions -falign-labels
        -falign-loops -falign-jumps -fexpensive-optimizations -ftree-vectorize"
CXXFLAGS="-O3 -funroll-loops -frerun-loop-opt -fschedule-insns2
          -frerun-cse-after-loop -falign-functions -falign-labels
          -falign-loops -falign-jumps -fexpensive-optimizations -ftree-vectorize"
\end{verbatim}

Additional gains are also possible through the activation of OpenMP
parallelism extensions.
