\documentclass[11pt]{article}

\def\xcolorversion{2.00}

\usepackage[version=latest]{pgf}

\usepackage{xkeyval,calc,listings,tikz}

% We need lots of libraries...
\usetikzlibrary{%
  arrows,%
  calc,%
  fit,%
  patterns,%
  plotmarks,%
  shapes.geometric,%
  shapes.misc,%
  shapes.symbols,%
  shapes.arrows,%
  shapes.callouts,%
  shapes.multipart,%
  shapes.gates.logic.US,%
  shapes.gates.logic.IEC,%
  er,%
  automata,%
  backgrounds,%
  chains,%
  topaths,%
  trees,%
  petri,%
  mindmap,%
  matrix,%
  calendar,%
  folding,%
  fadings,%
  through,%
  positioning,%
  scopes,%
  decorations.fractals,%
  decorations.shapes,%
  decorations.text,%
  decorations.pathmorphing,%
  decorations.pathreplacing,%
  decorations.footprints,%
  decorations.markings,%
  shadows}

\usepackage[a4paper,left=2.25cm,right=2.25cm,top=2.5cm,bottom=2.5cm,nohead]{geometry}
\usepackage{amsmath,amssymb}
\usepackage{xxcolor}
\usepackage{hyperref}

\newcommand{\os}{\textbf{OS} }

\newcommand{\linux}{\textbf{Linux} }
\newcommand{\ubuntu}{\textbf{Ubuntu} }
\newcommand{\debian}{\textbf{Debian} }

\begin{document}
  \title{Understanding Linux packaging}

  \author{Thomas Moulard}
  \maketitle

  \pagestyle{empty} %
  \thispagestyle{empty}

  \tableofcontents
  \newpage

  \section{Introduction}
  \label{intro}
  This article is trying to fulfill two different goals: firstly giving
an overview to \linux end-users how software is managed on their
computer and secondly explaining how to install or remove software
properly. If one force of \linux is its modularity and the huge amount
of alternatives to replace any system's subsystem, it is also very
difficult for newcomers to get a ``big picture'' of their system.

The following points will be discussed in this document:
\begin{itemize}
\item An overview of how \linux packaging is done
\item How to install or remove software using a tarball or a package,
\end{itemize}

  \section{A zoology of packaging systems}
  \label{zoology}
  % Tiks styles and settings.
\tikzset{
  pkgstate/.style={
    % The shape:
    rectangle,
    % The size:
    minimum size=6mm,
    % The border:
    very thick,
    draw=red!50!black!50,         % 50% red and 50% black,
                                  % and that mixed with 50% white
    % The filling:
    top color=white,              % a shading that is white at the top...
    bottom color=red!50!black!20, % and something else at the bottom
    % Font
    font=\itshape
  },
  pkgaction/.style={
    % The shape:
    rounded rectangle,
    minimum size=6mm,
    % The rest
    very thick,draw=black!50,
    top color=white,bottom color=black!20,
    font=\ttfamily},
  skip loop/.style={to path={-- ++(0,#1) -| (\tikztotarget)}}
}
\tikzset{pkgaction/.append style={text height=1.5ex,text depth=.25ex}}
\tikzset{pkgstate/.append style={text height=1.5ex,text depth=.25ex}}



To understand how packaging works, one has first to think about how
the work is shared between software developpers and \linux
distributions teams.  Software developpers are writing a specific
piece of software: they want it to be as broadly available as possible
(on different \os and on differents architectures). They will be
choosing develoment tools such as a compiler that will limit the
portability of their application but no other knowledge of their user
\os is available for them. Developpers will just make code as generic
and customizable as possible in order to make it work everywhere.  On
the opposite, maintainers are writing packages for a specific \linux
distribution: they do not target other systems and their distribution
is usually providing support for a limited number of computer
architectures. They know perfectly well how is organized their users
system and want to make all the software ecosystem working together as
best as possible.


This is why there exists two kinds of packages under \linux:
\begin{description}
\item[Tarballs] provided by software developpers,
\item[Distribution specific packages] provided by \linux distribution.
\end{description}


The second category of packages is always generated from te first: a
package is the tarball plus some additional tweaks, fine tuning to
adapt an application to a distribution and meta-information such as
the software name, author names\ldots

Packages can then be divided in two categories: source pacakges and
binary packages. Source packages has to be processed to be useful, it
contains sources files that can be used to generate the application
but not the application itself. On the opposite a binary package
directly contain the application and no further processing is
required.


For a given software a tarball is always available: it is the usual
way of publishing an application. On the opposite, providing a package
needs an additional effort which is not always made. This work is
usually done by volunteers that appreciate a project and thus
encourage its diffusion on their own distribution. Software popularity
is usually related to the number and quality of packages available.

\begin{figure}[htbp]
\centering
\begin{tikzpicture}[point/.style={coordinate},>=stealth',thick,draw=black!50,
                    tip/.style={->,shorten >=1pt},every join/.style={rounded corners},
                    hv path/.style={to path={-| (\tikztotarget)}},
                    vh path/.style={to path={|- (\tikztotarget)}}]
  \matrix[column sep=4mm] {
    % First row:
    & & & & & & & \node (pkgu) [pkgaction]    {package};
    & \node (ubuntu) [pkgstate] {Ubuntu package};\\
    % Second row:
    \node (p1) [point] {}; & \node (us)  [pkgstate] {unpublished software}; &
    \node (p2) [point] {}; & \node (rel) [pkgaction]    {release};          &
    \node (p3) [point] {}; & \node (tb)  [pkgstate] {tarball};\\
    % Third row:
    & & & & & & & \node (pkgf) [pkgaction]    {package};
    & \node (fedora)[pkgstate] {Fedora package};\\
  };

  { [start chain]
    \chainin (p1);
    \chainin (us)  [join=by tip];
    \chainin (rel) [join=by tip];
    \chainin (tb)  [join=by tip];
    { [start branch=ubuntu]
      \chainin (pkgu)   [join=by {vh path,tip}];
      \chainin (ubuntu) [join=by tip];
    }
    { [start branch=fedora]
      \chainin (pkgf)   [join=by {vh path,tip}];
      \chainin (fedora) [join=by tip];
    }
  }
\end{tikzpicture}
\caption{Packaging workflow}
\label{fig:software_workflow}
\end{figure}

As illustrated by \autoref{fig:software_workflow}, providing software
specific packages is the last step of the packaging: it needs
additional effort but it is much more convenient for the user: on
modern distributions, a single command-line or click can install,
upgrade or remove the software.


\subsection{Tarballs}

Project development is divided in different steps: those steps can
differ depending on the developpers policies but can be categorized in
three categories:
\begin{description}
\item[Initial development] is the steps where features are added to
  the software. During this phase, it is not recommended for users to
  rely on the application as many crashes are likely to occur.
\item[Stabilization] begins where enough features have been added to
  make the software useful. Developpers stop adding features and try
  to fix bugs and crashes.
\item[Test] is the phase where initial releases are done. They are
  usually named as ``alpha'' (when features are missing) then ``beta''
  (when bugs happens but most of the features are available) and
  finally ``release candidate or RC''. Early adopters then try these
  releases and give feedback to the development team. Depending on the
  software usability, the alpha becomes a beta and then a release
  candidate. Each release candidate is a potential definitive version,
  but if an important problem is detected, then the another RC release
  is done. This process is repeated as long as severe problems occur.
\item[Final release] is the last step. When developpers and users
  agree, a new stable version is released. This is the one that will
  be used by most of the users.
\end{description}

This description is just a rough idea of how release are working and
vary a lot from a project to another. The most important point is that
as long as there is no stable release, no packaging is done.


Software applications are procuded by writing text files called
``source code'' following a particular syntax called ``programming
language'' that are compiled to produce binaries (i.e. applications).

This applications also make use of external non-generated files such
as documentation or images for instance.

To finish, a software project also contain set of rules that describe
how to compile the project and where to install the different files on
the user's system.


A tarball is only a compressed archive containing these elements. Two
widely used formats are ``.tar.bz2'' and ``.tar.gz''.



Regarding end users, installing a software from a tarball is complex:
you will need the tools required to compile the project and some
fine-tuning will have to be made by hand. However, the control on the
software is complete: one can control what features will be enabled or
not, where the software will be installed\ldots Compiling yourself the
application can also increase application speed.


To finish, there is no meta-information in a tarball: one cannot get a
list of installed packages from a tarball and automatic updates are
impossible.


\subsection{Source packages}


Source packages are meta-information added to a tarball to make it
usable by a packaging system.
The added meta-information are typically:
\begin{description}
\item[Package tracking] Human readable package title, description,
  authors list, licensing information and web site URL.
\item[Compile-time dependencies] Tools used by the compilation process
  but not required when the application is runned.
\item[Run-time dependencies] Dependencies that have to remain
  available on the system each time the package is used such as shared
  libraries.
\item[Actions] A list of actions to perform at install, remove,
  updated (such as adding/removing a shortcut to the desktop).
\item[Patches] A set of modifications to perform on the software
  before compiling it in order to fix bugs and tune it for the current
  \linux distribution.
\end{description}


The packaging tool then stores all the meta-information in a
database. It allows automatic installation, update and removing.

One advantage of source packaging is that the user can choose the
installation directory and still has a good control of how the
software will be compiled without doing it manually.  The drawback is
that compile tools have to be installed and the compile time is very
long for complex software (many hours for large desktop applications).


\subsection{Binary packages}


Binary packaging is more and more popular as it is fully automatic: no
control is available for the user but it is very simple to use and
installation is almost instantaneous unlike source packaging. However,
it needs more effort than source packaging and is not always
available.


If source packaging adds meta-information to the tarball (i.e. to the
non-compiled software), a binary package adds meta-information to the
compiled program.

The packaging system install all the files in the same prefix (usually
\texttt{/usr}) so the user can not choose where the
software will be installed. The application is compiled once by the
package maintainer in the pre-determined installation directory.
However, the maintainer installs the application in a sandbox: an
empty temporary directory.  This directory will contain all the files
that would have been copied in the system. The content of the sandbox
directory and the meta-information file are archived and compressed.
This compressed archives is the binary package.


These source or binary packages are then often gathered in
``repositories'' to allow automatic retrieving from the package
management software installed on the user's computer.  A repository
contains a packages list and the corresponding files: each
distribution provide one or more default repositories but custom ones
also exist for minor packages or special versions (unstable or
dedicated to a particular task).


\subsection{Comparison}

The \autoref{fig:package_comparison} illustrates the advantages and
drawbacks of each packaging style. As expected, the tarball is the
most complex way of installing software and is not recommanded to
non-technical users. On the opposite, advanded users are often
interested to use the new versions of their software before the
packaging work is done.


On the opposite, source and binary packages allows easy installation
but as the settings are defined by the maintainers, the user does not
control anymore the fine tuning. For instance, experimental features
that are considered too unstable might be irreversably disabled by the
maintainers to ensure the qualitry of their \linux
distribution. Modifications are also done to better fit the
distribution: the location of configuration files are often changed to
match the distribution standard. It can lead to confusion new users
that read documentation on the web that do no match their system
package.


The last problem is package relocability: a package is said to be
relocable if the end-user can install it anywhere. However, only a few
package system support this: most of them force the installation to
\texttt{/usr}. This directory is only writeable by the
super-user (i.e. root user) which means that package management has to
be done by the computer administrator. If one is not the administrator
of its system, the only choice still remains the installation from the
tarball.

\begin{figure}[htbp]
\centering

\begin{tabular}[htbp]{|c|c|c|c|}
  \hline
  Packaging style & \textbf{Tarball} & \textbf{Source package} & \textbf{Binary package}\\
  \hline
  \emph{Dependency tracking}
  & \textcolor{red}{no}
  & \textcolor{green}{yes}
  & \textcolor{green}{yes}
  \\
  \emph{Enable/disable optional features}
  & \textcolor{green}{yes}
  & \textcolor{green}{yes}
  & \textcolor{red}{no}
  \\
  \emph{Package removing}
  & \textcolor{red}{unsafe}
  & \textcolor{green}{safe}
  & \textcolor{green}{safe}
  \\
  \emph{Installation time}
  & \textcolor{red}{long}
  & \textcolor{red}{long}
  & \textcolor{green}{short}
  \\
  \emph{Automatic package management}
  & \textcolor{red}{no}
  & \textcolor{green}{yes}
  & \textcolor{green}{yes}
  \\
  \emph{Installation choice}
  & \textcolor{green}{yes}
  & \textcolor{green}{yes}
  & \textcolor{red}{no}
  \\
  \emph{Root access}
  & \textcolor{green}{not needed}
  & \textcolor{red}{required}
  & \textcolor{red}{required}
  \\
  \emph{Ease of use}
  & \textcolor{red}{-}
  & \textcolor{green}{+}
  & \textcolor{green}{+}
  \\
  \emph{Package control}
  & \textcolor{green}{++}
  & \textcolor{green}{+}
  & \textcolor{red}{-}
  \\
  \hline
\end{tabular}

\caption{Comparison of packaging methods}
\label{fig:package_comparison}
\end{figure}



  \section{What is a software package?}
  \label{package}
  \subsection{Overview}

A software package contains all the required file to produce an
application or a library.


The content of the tarball is directly the unprocess material used by
software developpers to write software: it has to be processed to be
of use for an end-user. On the opposite, a binary package is of no use
for developpers but contain a directly usable application.


This section will discuss what are the different files present in a
package, how they are processed and where they are copied in the
directory hierarchy.


\begin{description}
\item[source files] are files that can be compiled to produce an
  executable file.
\item[resources] are text or binary files used by an application
  (i.e. images, icons, \ldots).
\item[documentation] can be copied or generated depending on its
  format.
\end{description}


\subsection{Source files}


Many programming languages exist and each has its own building
mechanism, the goal of this paragraph is only giving a general idea
about how the building process is generally working.


Programming languages can be categorized in compiled or interpreted
languages. Compiled languages such as C, C++, Pascal transform source
files (text files containing instructions) into an executable. If the
executable has an entry point, it is a binary which can be runned by
the user, if there is not it is a library. Libraries are designed to
store sets of procedures that will be reused by different
applications.  The other kind of programming languages are interpreted
languages such as Python, Ruby: the end-user has to run an interpreter
on a source file itself to run the program. It is repeated each time
the software is used.
The compiler is a compile-time dependency whether an interpreter is a
run-time dependency.


\begin{figure}[htbp]
\centering
\begin{tikzpicture}[point/.style={coordinate},>=stealth',thick,draw=black!50,
                    tip/.style={->,shorten >=1pt},every join/.style={rounded corners},
                    hv path/.style={to path={-| (\tikztotarget)}},
                    vh path/.style={to path={|- (\tikztotarget)}}]
  \matrix[column sep=4mm] {
    \node (source)   [pkgstate]    {source file};\\
    & \node (compiler) [pkgaction]   {compiler}; &
    \node (binary)   [pkgstate]    {binary or library};\\

    \node (source2)   [pkgstate]    {source file};\\
  };

  { [start chain]
    { [start branch]
      \chainin (source);
      \chainin (compiler) [join=by {vh path,tip}];
    }
    { [start branch]
      \chainin (source2);
      \chainin (compiler) [join=by {vh path,tip}];
    }
      \chainin (compiler);
    \chainin (binary)  [join=by tip];
  }
\end{tikzpicture}
\caption{Compiling workflow}
\label{fig:compiling_workflow}
\end{figure}


\subsection{Resources}

Resources are the set of all files that are just bundled with an
application without having any transformation applied on them. It
gathers multimedia files such as images, video and sounds but also
preference files.

If the majority are just copied, some can also be generated by the
packaging process to match more precisely the host system.


\subsection{Documentation}

Documentation in software have several forms, especially on
\linux. Each application should provide a \texttt{man} (manual) page
which describes the program usage (its goal and the command line
options it accepts). A more structured format are \texttt{info} files
that contain the whole reference manual and is usually much longer.


However these formats are designed to be viewed in console and are
hence quite limited: no images, sounds or video, no special fonts or
complex type setting. Other documentation systems such as \LaTeX or
HTML/XML based formats such as Docbook are more and more used.


All these formats however share one common point: the document is
generated from a source file that describe the document.


The last case is self-documentation of the source code: this kind of
documentation is generated from tools such as Doxygen which analyzes
the source code from a package and generates developper-oriented
documentation.


\subsection{Building mechanisms}

As seen in the previous sections, different kind of processing has to
be done to generate a project. These processes are controled by
\texttt{Makefile} files. The \texttt{Makefile} format defines
generation rules, i.e. its links a particular kind of output with its
input and shell command lines. For instance, one can define what
command lines have to runned to get a binary from source files. It is
also able to detect what has to be regenerated when changes are done
to source files.


However, these rules depends on plenty of parameters. Some of them
depends on the host platform, on the \linux distribution or even the
user setup! This is the reason a \texttt{configure} file is needed:
this tool tunes the \texttt{Makefile} files in order to adapt them to
the host system.

It controls:
\begin{description}
\item[Compiler] What compiler will be used as several compiler exist
  for most of the programming languages.
\item[Directories] Where will the software be installed on the host
  system?
\item[Compilation profile] What flags will be given to the compiler:
  it will determine if debugging support will be included, if the
  application will be optimized, \ldots
\item[Enabling of optional features] What features will be included in
  the final software?
\end{description}

It also realize severy sanitary checks to make sure all dependencies
(tools, libraries\ldots) are present and that versions are newer
enough.

  \section{Installing a software}
  \label{install}
  \subsection{\linux Directories overview}

The organization of a \linux file system is completely defined by the
\emph{Filesystem Hierarchy Standard}.

It is defining the notion of \emph{prefix}. A prefix is a directory
that will contain the installation of one or more software,
libraries\ldots
It contains several standardized sub-directories such as:
\begin{description}
  \item[\texttt{bin}] for executable binaries (i.e. applications),
  \item[\texttt{etc}] for editable configuration files,
  \item[\texttt{include}] for header files (C/C++ specific).
  \item[\texttt{lib}] for static and shared libraries,
  \item[\texttt{share}] for resources. In particular, documentation is\
    stored in \texttt{share/doc}.
\end{description}

Some standard prefixes are managed by the system:
\begin{description}
\item[\texttt{/}] Root prefix, containt low-level software required
  for computer boot.
\item[\texttt{/usr}] stands for Unix System Resources, contains
  software managed by the local packaging system.
\item[\texttt{/usr/games}] for games.
\item[\texttt{/usr/local}] is the default prefix for manually managed
  software. It means that an automatic packaging system will
  \emph{never} modify the content of this directory.
\end{description}

These prefixes are shared: any user can access to them with the
exception of the \texttt{/usr/games} prefix which is owned by the
\texttt{games} group. It means that one has to be part of this group
to run a game.


Being a prefix is an absolutely arbitrary and symbolic decisions:
there is no list of prefixes on the system and a prefix is a
``normal'' directory. Only its structure can indicate that a directory
is a prefix.

It also means that a user can create a new prefix to install some
packages separately. Several reasons can lead to this choice such as
having the possibility to install development or unstable version of
some software. As no non-shared prefix is available by default, users
who do not have root access to their machine have to setup a local
prefix in their home directory to root access.


\subsection{Adding a prefix manually}

Creating a prefix is just adding some directories. Int this section, a
prefix will be created in the \texttt{/my/new/prefix} directory. Feel
free to replace it by whatever you want in each command.

\begin{verbatim}
$ mkdir -p /my/new/prefix  && \
  mkdir /my/new/prefix/bin && \
  mkdir /my/new/prefix/lib
\end{verbatim}

Then, fix directories permissions to give access to the users and
groups you want. Read-only access is sufficient to use software, write
access has to be given for people who will be able to manage software
in this prefix. Directories also have to be traverseable.


Additional steps are required so installed files will be found xby
system tools.

An environment variable is a variable which is passed to the shell
sub-processes (typically the application you want to launch).  Some of
them have special meanings and control the system behavior.

In particular the \texttt{PATH} environment variable contains the list
of the directories where binaries are stored. The binary directory of
the new prefix has to be prepended to the variable.

For the same reason, \texttt{LIBRARY\_PATH} and
\texttt{LD\_LIBRARY\_PATH} have to be updated with the new library
directories. The first one is used by the C/C++ compilers to locate
libraries whether the second is used to retrieve libraries at run-time
when an application is launched. \texttt{C\_INCLUDE\_PATH} adds the
\texttt{include} directory to the list of default header files.

\texttt{INFOPATH} and \texttt{MANPATH} are updated to indicate the new
documentation directories.

The environment variables modifications can be done through the
following command lines:

\begin{verbatim}
# If you use a Bourne shell:
$ export C_INCLUDE_PATH="/my/new/prefix/include:${C_INCLUDE_PATH}"
$ export LD_LIBRARY_PATH="/my/new/prefix/lib:${LD_LIBRARY_PATH}"
$ export LIBRARY_PATH="/my/new/prefix/lib:${LIBRARY_PATH}"
$ export INFOPATH="/my/new/prefix/share/info:${INFOPATH}"
$ export MANPATH="/my/new/prefix/share/man:${MANPATH}"
$ export PATH="/my/new/prefix/bin:${PATH}"

# If you use a C-shell:
$ setenv C_INCLUDE_PATH "/my/new/prefix/include:${C_INCLUDE_PATH}"
$ setenv LD_LIBRARY_PATH "/my/new/prefix/lib:${LD_LIBRARY_PATH}"
$ setenv LDPATH "/my/new/prefix/lib:${LIBRARY_PATH}"
$ setenv INFOPATH "/my/new/prefix/share/info:${INFOPATH}"
$ setenv MANPATH "/my/new/prefix/share/man:${MANPATH}"
$ setenv PATH "/my/new/prefix/bin:${PATH}"
\end{verbatim}

However, this commands do not change the value of an environment
variable permanently, it only change it in your local shell
session. To make the modification global, you have to add them to your
own shell configuration file (see your shell documentation for more
information).

To finish, keep in mind that the order in this directory list is
important: a list is parsed from left to right. It means that if one
tries to start a program called \texttt{foo}, the leftest prefix
containing that application will be used.


\subsection{Using distribution-specific packages (source or binary)}


To use a package, one has to launch the tool that are provided with
their \linux distribution. Adding or removing files in a managed
prefix can break permanently a system.


The packaging systems currently provide command line tools and graphic
tools. For instance on \ubuntu, the graphical tool is called
\texttt{synaptics} whereas the command line tools are \texttt{apt-*}
tools (\texttt{apt-get}, \texttt{apt-cache}, \ldots).


\subsection{Using tarballs}


Tarballs are much more difficult to install than packages as the
process has to be done manually.  The next paragraphs will describe
step by step how to use a tarball.


\subsubsection{Obtaining a tarball}

Most of the projects are providing a tarball using the
\texttt{.tar.bz2} or \texttt{.tar.gz} formats. Download it on the
project web page.

They sometimes provide a hashsum of the file to make sure no error
happened. It is recommended to check that the hashsum matches the
downloaded file.


\subsubsection{Uncompressing a tarball}


A tarball is a \textbf{compressed archive}. It is created by two
separate steps: archiving and compression.

Archiving is the process of taking several files and directories and
put them in a single file. This file is called an archive. The most
common archiver is \texttt{tar} and produced \texttt{tar} files
(i.e. \texttt{my-software.tar}\footnote{and not
  \texttt{my-software.tar.something}!}).

The compression is a second process which takes one or several files
and then compress it/them. Two popular algorithms and gzip and bzip
producing respectively \texttt{gz} files and \texttt{bzip} files.

Indeed, a \texttt{.tar.bz2} file is an archive that has been
compressed using the bzip algorithm.


To uncompress and unarchive a file, one can type:
\begin{verbatim}
# For a bzipped tarball:
$ tar xjvf my-software.tar.bz2
# For a gzipped tarball:
$ tar xzvf my-software.tar.gz
\end{verbatim}

A good and widely respected practice under \linux is that tarballs
contains a single directory with the same name as the archive
containing its whole content. It avoids mixing the tarball's contents
with the user's files.


\subsubsection{Configuring}


Once the tarball is uncompressed, user has to setup the package by
launching the \texttt{configure} script shell.

A wise idea is reading the \texttt{INSTALL}, \texttt{README} and
\texttt{NEWS} files to read about specific issues and last
modification of the software.

Then \texttt{./configure --help} will display the list of options for
the package. The standard options will be discussed here but specific
ones such as optional features will differ from pacakge to package.


It is generally a good idea to compile out of source. It means that
generated files will be stored in a separate directory instead of the
source directories. By doing it, one just have to change its build
directory to restart from scratch without error. Generated files which
are not deleted properly are the source of many compilation errors.

To build out of source, just choose a directory and call the
\texttt{configure} script from it. For instance, one can create a
\texttt{\_build} directory at the root-level of the uncompressed
package:

\begin{verbatim}
# To be runned in the package root directory:
$ mkdir _build && cd _build && ../configure
\end{verbatim}


However, one might want to customize the package, it can be done
through options or environment variables.


Several standard options are provided:
\begin{description}
\item[\texttt{--prefix}] The prefix that will be used for installation
\item[\texttt{--XXXdir}] Binary, resources directories\ldots can also
  be defined by hand instead of being its default prefix
  sub-directory.
\item[\texttt{--program-prefix}, \texttt{--program-suffix},
  \texttt{--program-transform-name}] are useful options used to
  transform program name. It is used to install several version of the
  same program without any conflict.
\item[\texttt{--host}] can be used for cross-compiling (i.e. compiling
  for a platform different from the computer running the compilation).
\end{description}


The environment variables control the behavior of the compiler. Here
is some tips for C/C++ packages which represent the larger part of the
software packages:

\begin{description}
\item[\texttt{CC}] defines the C compiler.
\item[\texttt{CXX}] defines the C++ compiler.
\item[\texttt{CPPFLAGS}] defines the flags that will be passed to the
  C/C++ pre-processor.
\item[\texttt{CFLAGS}] defines the flags that will be passed to the C
  compiler.
\item[\texttt{CXXFLAGS}] defines the flags that will be passed to the
  C++ compiler.
\item[\texttt{LDFLAGS}] defines the flags that will be passed to the
  linker.
\end{description}

Flags passed to the pre-processor and to the linker are usually only
custom header and libraries locations. On the opposite,
\texttt{CFLAGS} and \texttt{CXXFLAGS} controls how the compiler will
build the binaries.

Compilers accept a huge set of options which will not be described
here. Please consult your compiler documentation for the best set of
options depending on your goal (debug, release, packaging) and
platform. Some general purpose compilation profiles are available in
\autoref{compilation_profiles}.


\subsubsection{Building and installing}


Once the package is configured, you build directory contains
\texttt{Makefile} files that can be used to build the software.
Only one command line is required:

\begin{verbatim}
$ make
\end{verbatim}


It starts the package compilation process which can take up to several
hours. When it is finished, you can install the package through the
following command:

\begin{verbatim}
$ make install
# ...or if install is done in a directory the current user can not write in:
$ sudo make install
\end{verbatim}

It is important to \emph{not} launch the first \emph{make} command as
a super-user: it is unsafe (a bad or malicious package could damage
your system), unneeded and create object files owned by the root user
that will disturb your work if you have to re-use your build
directory.


To check the installation step without root access, one can install in a sandbox by using:
\begin{verbatim}
$ make install DESTDIR=/my/temporary/prefix
\end{verbatim}

It will simulate that the \texttt{DESTDIR} directory is the root
directory. It will create the path to your prefix \textbf{inside} and
install everything. If your prefix is \texttt{foo}, your binary
directory \texttt{foo/bin} and you are building a programm called
\texttt{bar}, it will install it in
\texttt{/my/temporary/prefix/foo/bin/bar}.  This feature is
interesting when building a binary package: one configure the software
using the prefix wanted by the packaging system (usually
\texttt{/usr}), compile and install the whole package in a
sandbox. You then get a snapshot of the build package which is very
easy to transform into a full binary package.


  \appendix
  \newpage

  \section{GCC compilation profiles}
  \label{compilation_profiles}
  GCC accept a large set of flags and it would be difficult to discuss them all here. On the opposite, there is only a few amount of user categories:
\begin{description}
\item[developpers] or power-users that want to get technical information,,
\item[maintainers] which want to package the software,
\item[end-users] that just want to use the package,
\end{description}

\subsection{Developper profile}

The developper profile:
\begin{itemize}
\item allow easy debugging by adding debugging symbols and not enabling optimization.
\item do not disable assertions.
\end{itemize}

Basically, this is almost completely the default behavior of GCC so very few options are required:
\begin{verbatim}
CFLAGS="-g -O0"
CXXFLAGS="-g -O0"
\end{verbatim}

If one only use GDB as a debugger, \texttt{-g} can be replaced by
\texttt{-ggdb} to enable GDB specific debugging extensions.

\subsection{End-user and maintainer profile}

The end-user profile:
\begin{itemize}
\item want to get the application to run as fast as possible,
\item does not want technical information
\item wants small binaries.
\end{itemize}

\begin{verbatim}
CPPFLAGS="-DNDEBUG"
CFLAGS="-O3"
CXXFLAGS="-O3"
\end{verbatim}

One can also add the \texttt{-march} options to \texttt{CFLAGS} and
\texttt{CXXFLAGS} to build binaries specific to your platform.  For
intance, to generate binaries optimized for Intel Core 2 Duo
processors, one can use \texttt{-march=core2 -msse4.1} (with
\texttt{GCC >= 4.3}).

To find the options matching your processor, see the compiler
documentation. Processor information can be displayed using:
\begin{verbatim}
$ cat /proc/cpuinfo
\end{verbatim}


The maintainer wants almost the same thing than the end-user except
that the binaries are built to be shared. It means processor-specific
flags must be avoided.


To finish, aggressive compilations flags can be used to optimize more
the binaries. Some of these flags are not supported by older versions
of GCC or may trigger problems or bugs:

\begin{verbatim}
CPPFLAGS="-DNDEBUG"
CFLAGS="-O3 -funroll-loops -frerun-loop-opt -fschedule-insns2
        -frerun-cse-after-loop -falign-functions -falign-labels
        -falign-loops -falign-jumps -fexpensive-optimizations -ftree-vectorize"
CXXFLAGS="-O3 -funroll-loops -frerun-loop-opt -fschedule-insns2
          -frerun-cse-after-loop -falign-functions -falign-labels
          -falign-loops -falign-jumps -fexpensive-optimizations -ftree-vectorize"
\end{verbatim}

Additional gains are also possible through the activation of OpenMP
parallelism extensions.


%  \nocite{*}
%  \bibliography{linux-packaging}
\end{document}
